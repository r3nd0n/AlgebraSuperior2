\documentclass[12pt]{article}
\usepackage{graphicx}
\usepackage{caption}
\usepackage{subcaption}
\usepackage{tikz}
\usepackage{venndiagram}
\usepackage{tcolorbox}
\usepackage{listings}
\usepackage{enumitem}
\usepackage{amsmath}
\usepackage{amssymb}
\usepackage{colortbl}
\usepackage{xcolor}
\usepackage[margin=1cm, top=1.5cm, bottom=1.5cm]{geometry}
\usepackage{fancyhdr} % Paquete para encabezados y pies de página

\tcbuselibrary{breakable}

\title{\textbf{Álgebra Superior II\\
		Tarea 02: Divisibilidad}}
\author{Mora Espinosa Miroslava\\ Rendón Ávila Jesús Mateo\\ Rubio Pérez Ángel Damián\\ Valencia Morales Indra Gabriel}
\date{\today}

%\pagestyle{fancy} % Establece el estilo de página como fancy
%\fancyhf{} % Limpia cualquier configuración previa de encabezado y pie de página
%\fancyhead[L]{Rendón Ávila Jesús Mateo} % Encabezado izquierdo
%\fancyhead[R]{Tarea 02: Relaciones y Funciones} % Encabezado derecho
%\fancyfoot[C]{\thepage} % Pie de página centrado con el número de página

\begin{document}
	\maketitle
	\begin{center}
		\vspace{2cm}
		\includegraphics[width=0.17\textwidth]{Escudo.png}
		\hspace{0.5cm}
		\includegraphics[width=0.2\textwidth]{logo_ciencias.png}
	\end{center}
	\begin{center}
		\vspace{1cm}
		Facultad de Ciencias\\
		Universidad Nacional Autónoma de México\\
		\vspace{3cm}
		Profesor: Dr. Gerardo Miguel Tecpa Galván\\
	\end{center}
	
	\newpage
	%
	% Ejercicio 1
	%
	
	$ \ $ \\
	$\bullet$\textbf{1.} Sean $a$ un número par y $b$ un número impar. Muestra que $mcd(a, b)$ es impar.
	\subsubsection*{Respuesta}
	Procedemos por contradicción. Supongamos que $mcd(a, b)$ es par.\\
	
	Por definición de $mcd$, entonces $mcd(a,b) \mid a$ y $mcd(a,b) \mid b$.\\
	
	Por ser $mcd(a,b)$ par, entonces $mcd(a,b) \nmid b$ \textbf{!}\\
	
	De lo anterior debe ser $mcd(a,b)$ es impar.$_\blacksquare$
	
	\vspace{1cm}
	
	%
	% Ejercicio 2
	%
	$\bullet$ \textbf{2.} Un grupo de 23 viajeros llega a un campamento y encuentra 63 montones de sacos, cada montón con el mismo número de sacos, y un montón adicional con 7 sacos (en total hay 64 montones). Si sabemos que los viajeros no podían cargar con más de 50 sacos cada uno y pudieron repartírselos por igual y sin abrirlos, ¿cuántos sacos había en cada uno de los montones?
	
	\subsubsection*{Respuesta}
	Como un viajero puede llevar a lo mas 50 sacos y hay $x$ sacos en 63 montones y 7 sacos 
	sueltos, podemos obtener lo siguiente:
	
	\[23 \mid 63x + 7\]
	\[\textit{i.e } 63x \equiv -7 \textit{ mod } 23\]
	Propongamos $x = 5$ tendriamos entonces:
	\[23 \mid 63 \cdot 5 + 7\]
	\[23 \mid 315 + 7\]
	\[23 \mid 322\]
	Por definición de divisibilidad $Existe \ \ast \in \mathbb{N}$ tal que $23 \cdot \ast = 322$.\\
	
	Si $\ast = 14$, entonces $23 \cdot 14 = 322$\\
	
	Así, concluimos que habia 5 sacos por monton $_\blacksquare$
	
	\vspace{1cm}
	
	%
	% Ejercicio 3
	%
	$\bullet$ \textbf{3.} Demuestra que si $a$ y $b$ son enteros no nulos, entonces $mcd(a, b) |mcm(a, b)$.
	
	\subsubsection*{Respuesta}
	Sean a y b enteros no nulos\\
	Sabemos que por definicion de mínimo comun múltiplo\\
	$a \mid mcm(a,b)$ y $b \mid mcm(a,b)$\\
	De igual manera, sabemos que por definición de maximo común divisor\\
	$mcd(a,b) \mid a$ y $mcd(a,b) \mid b$\\
	\\
	Por transitividad de la divisibilidad.\\
	Como $mcd(a,b) \mid a$ y $a \mid mcm(a,b)$, entonces $mcd(a,b) \mid mcm(a,b)$\\
	Como $mcd(a,b) \mid b$ y $b \mid mcm(a,b)$, entonces $mcd(a,b) \mid mcm(a,b)$\\
	\\
	$\therefore$ $mcd(a, b) |mcm(a, b)$. $_\blacksquare$
	
	\vspace{1cm}
	
	%
	% Ejercicio 4
	%
	$\bullet$ \textbf{4.} Muestra que si $p$ y $q$ son dos primos distintos, entonces para todo $a, b \in \mathbb{Z}^+$ se cumple que\\
	$mcd(p^a, q^b) = 1$.
	\subsubsection*{Respuesta}
	%Sabemos que $p$ es primo, enotnces $p \geq 2$ y solo pasa que $1 \mid p$ y $p \mid p$\\
	%Sabemos que $q$ es primo, enotnces $q \geq 2$ y solo pasa que $1 \mid q$ y $q \mid q$\\
	%Como $p$ y $q$ son primos y sabemos que $p \neq q$, enotnces $mcd(p,q) = 1$\\
	%Tomando un $a \in \mathbb{Z}^+$ tal que $p^{a}$ podmeos verlo como $p_1 \cdot p_2 \cdots p_a$, $p_i = p_j$ con $i,j \in \{1,2, \dots, a\}$\\
	%igualmente $b \in \mathbb{Z}^+$ tal que $q^{b}$ podemos verlo como $q_1 \cdot q_2 \cdots q_b$, $q_i = q_j$ con $i, j \in \{1,2, \dots, b\}$\\
	%Podemos ver que $q_1 \cdot q_2 \cdots q_b$ y $p_1 \cdot p_2 \cdots p_a$ no tienen factores primos en común.\\
	%De lo anteiror, deber ser $mcd(p^a, q^b) = 1$ $_\blacksquare$
	Sabemos que $p$ y $q$ son primos y ademas $p \neq q$\\

	Sean entonces $a, b \in \mathbb{Z}^+$ tal que $mcd(p^a, q^b) \neq 1$ y $mcd(p^a, q^b) = r$\\

	Por definición de $mcd$, entonces:
	\[r \mid p^a \textit{ y } r \mid q^b\]
	
	De $r \mid p^a$ podemos concluir que $r \mid p$\\

	De $r \mid q^b$ podemos concluir que $r \mid q$\\

	Pero sabemos que $p \neq q$, por lo que debe ser $mcd(p^a, q^b) = 1$ \textbf{!}\\

	Por lo tanto debe ser $mcd(p^a, q^b) = 1$ $_\blacksquare$
	
	\vspace{1cm}
	
	%
	% Ejercicio 5
	%
	$\bullet$ \textbf{5.} Sean $a_1, \dots, a_n \in \mathbb{Z}$ una colección de enteros Muestra que si para todo $i, j \in \{1, . . . , k \}$ 
	con $i \neq j$ se satisface que $mcd(a_i, a_j ) = 1$, entonces $mcd(a_k , a_1  \cdot a_2 \cdots a_{k-1}) = 1$.
	
	\subsubsection*{Respuesta}
	Procedemos por inducción.\\
	
	Sea $p$ primo, por ser primo $p \geq 2$ y supongamos que $p \mid mcd(a_k , a_1  \cdot a_2 \cdots a_{k-1})$.
	y sea $a_l$ con $l \in \{1, \dots, k -1 \}$\\
	
	Por definición enotnces $mcd(a_k , a_1  \cdot a_2 \cdots a_{k-1}) \mid a_k$ y $mcd(a_k , a_1  \cdot a_2 \cdots a_{k-1}) \mid a_l$.\\
	
	Así $p \mid a_k$ y $p \mid a_l$. Por definición de $mcd$, entonces $p \leq mcd(a_k, a_l)$.\\
	
	Por hipótesis sabemos que $mcd(a_k, a_l) = 1$ \textbf{!}\\
	
	Como no puede ser $p > 1$ y $p \leq 1$, entonces debe ser $mcd(a_k , a_1  \cdot a_2 \cdots a_{k-1}) = 1$ $_\blacksquare$
	
	\vspace{1cm}
	
	%
	% Ejercicio 6
	%
	$\bullet$ \textbf{6.} Sean $a_1, \dots, a_n, b \in \mathbb{Z}$ una colección de enteros tales que para todo $i, j \in \{1, \dots, k\}$ con $i \neq j$
	se satisface que $mcd(a_i, a_j ) = 1$. Muestra por inducción que si para todo $i \in \{1, \dots, k \}$ se cumple que $a_i |b$, entonces
	$a_1 \cdot a_2 \cdots a_k |b$.
	
	\subsubsection*{Respuesta}
	Mostraremos mediante inducción matematica que si para todo $i \in \{1, \dots, k \}$ se cumple que $a_i \mid b$, entonces $a_1 \cdot a_2 \cdots a_k \mid b$.\\
	\\
	\textbf{Base de inducción:} Mostraremos para k=2 que si para todo $i \in \{1,2\}$ se cumple que $a_i \mid b$, entonces $a_1 \cdot a_2 \mid b$.\\\\ 
	Sean $a_1, a_2, b \in \mathbb{Z}$ una colección de enteros tales que para todo $i, j \in \{1, 2\}$ con $i \neq j$ se satisface que $mcd(a_i, a_j ) = 1$ .\\\\
	Como $i, j \in \{1,2\}$, entonces se cumple que $a_i\mid b$ y $a_j \mid b$.\\\\
	Tomemos $i=1$ y como $i \neq j$ sea $j=2$.\\\\
	Como $a_1 \mid b$, $a_2 \mid b$ y por hipótesis  $mcd(a_i, a_j ) = mcd(a_1,a_2)= 1$, por lema:\\\\
	$\therefore$ $a_1 \cdot a_2 \mid b $\\\\
	$\therefore$ Se cumple el enunciado para la base inductiva. $ _{\blacksquare }$\\\\
	\textbf{Hipotesis de inducción:} Supongamos para k= n que si para todo $i \in \{1, \dots, n \}$ se cumple que $a_i \mid b$, entonces $a_1 \cdot a_2 \cdots a_n \mid b$\\\\
	\textbf{Paso de inducción:} Mostraremos para k=n+1 que si para todo $i \in \{1, \dots, n+1 \}$ se cumple que $a_i \mid b$, entonces $a_1 \cdot a_2 \cdots a_n \cdot a_{n + 1} \mid b$\\\\
	Sean $a_1, \dots, a_n, a_{n+1}, b \in \mathbb{Z}$ una colección de enteros tales que para todo $i, j \in \{1, ...,n, n+1\}$ con $i \neq j$ se satisface que $mcd(a_i, a_j ) = 1$, por el ejercicio \textbf{5} tenemos que $mcd(a_{n+1}, a_1 \cdot a_2 \cdots a_n) = 1$\\\\
	Veamos que $mcd (a_1 \cdot a_2 \cdots a_n, a_{n+1})=mcd(a_{n+1}, a_1 \cdot a_2 \cdots a_n)$ por lema, entonces $mcd (a_1 \cdot a_2 \cdots a_n, a_{n+1})= 1$ \\\\
	Además, como  $i, j \in \{1, ...,n, n+1\}$, entonces $a_i \mid b$ y $a_j\mid b$.\\\\
	Notemos que $i \in \{1,..,n\}$ y además se cumple que $a_i \mid b$, entonces por hipótesis de inducción $a_1 \cdot a_2 \cdots a_n \mid b$\\\\
	Como $j \in \{1, ...,n, n+1\}$, $a_j \mid b$ e $i \neq j$, en particular $a_{n+1} \mid b$.\\\\
	Como $mcd(a_1 \cdot a_2 \cdots a_n, a_{n+1}) = 1$,  $a_1 \cdot a_2 \cdots a_n \mid b$ y $a_{n+1} \mid b$, por lema:\\\\
	$\therefore$ $a_1 \cdot a_2 \cdot  a_n \cdot a_{n + 1} \mid b$ \\\\
	$\therefore$ Se cumple el enunciado para el Paso inductivo.\\\\
	$\therefore$ \textbf{Si para todo $i \in \{1, \dots, k \}$ se cumple que $a_i |b$, entonces
		$a_1 \cdot a_2 \cdots a_k |b$. $_\blacksquare$}
	
	\vspace{1cm}
	
	%
	% ejercicio 7
	%
	$\bullet$ \textbf{7.} Sea $p$ un número primo. Muestra que si $k \in \mathbb{Z}$ es tal que $k < p$, entonces $p \nmid k!$
	
	\subsubsection*{Respuesta}
	Supongamos por contradicción que $p \ | \ k!$ , es decir
	
	\[
	p \ | \prod_{i=1}^{k} i
	\]
	entonces, existe $m \in \{1, ... , k\}$ tal que $p \ | \ m$\\
	Pero, como $k<p$ y $m<k$, entonces $m<p$, pero eso implica que $p \nmid m$\\
	ya que no existe $w \in \mathbb{Z}$ tal que $wp = m$\\
	$\therefore p \nmid m \ \textbf{   !}$ \\
	$\therefore p \nmid k!$ $ _{\blacksquare }$
	
	\vspace{1cm}
	
	%
	% Ejercicio 8
	%
	$\bullet$ \textbf{8.} Sean $c \neq 0$ y $k \geq 2$. Muestra mediante inducción matemática que si $a_1, \dots, a_k$ es una colección de enteros no nulos, entonces $mcm(ca_1, ca_2, \dots, ca_k ) = |c|mcm(a_1, a_2, \dots, a_k )$.
	
	\subsubsection*{Respuesta}
	Mostraremos mediante inducción matematica que si $a_1,...,a_k$ es una colección de enteros no nulos, entonces $mcm(ca_1, ca_2, \dots, ca_k ) = |c| \cdot mcm(a_1, a_2, \dots, a_k )$. \\
	\\
	\textbf{Caso base}: Probaremos para k=2 y $c \neq 0$, que si $a_1, a_2$ es una colección de enteros no nulos, entonces  $mcm(ca_1, ca_2) = |c| \cdot mcm(a_1, a_2)$\\
	\\
	Sea  $a_1, a_2$ una colección de enteros no nulos, entonces notemos que
	
	\begin{table}[h]
		\begin{tabular}{lcrp{6cm}}
			$mcm (ca_1, ca_2)$ & $=\frac{|ca_1 \cdot ca_2|}{mcd(ca_1,ca_2)}$ & Por teorema \\
			\\
			\ \ & $=\frac{|c^2 a_1 \cdot a_2|}{mcd(ca_1,ca_2)}$ & Por aritmética \\
			\\
			\ \ & $=\frac{|c^2| |a_1\cdot a_2|}{|c| \cdot mcd(a_1,a_2)}$ & Por propiedad de valor absoluto\\
			\\
			\ \ & $=\frac{|c| \cdot |c| \cdot |a_1\cdot a_2|}{|c| \cdot mcd(a_1,a_2)}$ & Por propiedad de valor absoluto\\
			\\
			\ \ & $= |c| \cdot \frac{|a_1\cdot a_2|}{mcd(a_1,a_2)}$ & Por hipótesis $c \neq 0$ y por prop de mcd \\
			\\
			\ \ & $=|c| \cdot mcm(a_1,a_2)$ $_\blacksquare$ & Por teorema \\
			
		\end{tabular}
	\end{table}
	
	$\therefore$ $mcm(ca_1, ca_2) = |c| \cdot mcm(a_1, a_2)$, el caso base se cumple. \\
	\\
	\textbf{Hipótesis inductiva}: Supondremos para k=n y $c \neq 0$, que si $a_1,..., a_n$ es una colección de enteros no nulos, entonces  $mcm(ca_1, ..., ca_n) = |c| \cdot mcm(a_1, ..., a_n)$\\
	\\
	\textbf{Paso inductivo}: Mostraremos para $k=n+1$ y $c \neq 0$, que si $a_1,..., a_n, a_{n+1}$ es una colección de enteros no nulos, entonces $mcm(ca_1, ..., ca_n,a_{n+1}) = |c| \cdot mcm(a_1, ..., a_n, a_{n+1})$\\
	\\
	Sea $a_1,..., a_n, a_{n+1}$ una colección de enteros no nulos, en particular,\\
	$a_1,..., a_n$ es una colección de enteros no nulos, entonces $mcm(ca_1, ..., ca_n)$ $=|c| \cdot mcm(a_1, ..., a_n)$ por hipótesis de inducción. 
	\begin{table}[h]
		\begin{tabular}{lcrp{6cm}}
			Así $mcm(ca_1, ..., ca_n, ca_{n+1})$ & $=mcm(mcm(ca_1,..., ca_n),ca_{n+1})$ & Por teorema\\
			\\
			\ \ & $=mcm(|c| \cdot mcm(a_1,..., a_n),ca_{n+1})$ & Por hip. inductiva\\
			\\
			\ \ & $=\frac{ \mid (|c| \cdot mcm(a_1,...,a_n))\cdot ca_{n+1} \mid }{mcd(|c| \cdot mcm(a_1,...,a_n),ca_{n+1})}$ & Por teorema\\
			\\
			\ \ & $=\frac{ \mid c \cdot mcm(a_1,...,a_n) \cdot ca_{n+1} \mid }{mcd(|c| \cdot mcm(a_1,...,a_n),ca_{n+1})}$ & Simplificando \\
		\end{tabular}
	\end{table}
	
	\begin{tcolorbox}
		[width=\linewidth,sharp corners=all, colback=white!95!black]
		\textbf{Observación}
		
		Notemos que para $mcd(|c| \cdot mcm(a_1,...,a_n),ca_{n+1}$:
		\begin{itemize}
			\item Si $c<0$ entonces\\
			$mcd(|c|mcm(a_1,...,a_n),ca_{n+1})=mcd((-c) \cdot mcm(a_1,...,a_n),ca_{n+1})$, pero por propiedad de máximo común divisor, $mcd((-c) \cdot mcm(a_1,...,a_n),ca_{n+1}) = mcd(c \cdot mcm(a_1,...,a_n),ca_{n+1})$\\
			\\
			Por hipotesis como $c \neq 0$, entonces\\
			$mcd(c \cdot mcm(a_1,...,a_n),ca_{n+1})$ = $|c| \cdot mcd(mcm(a_1,...,a_n),a_{n+1})$ por lema.
			
			\item Si $c>0$ entonces\\
			$mcd(|c|mcm(a_1,...,a_n),ca_{n+1})=mcd(c \cdot mcm(a_1,...,a_n),ca_{n+1})$\\
			\\
			Por hipotesis como $c \neq 0$, entonces\\
			$mcd(c \cdot mcm(a_1,...,a_n),ca_{n+1})$ = $|c| \cdot mcd(mcm(a_1,...,a_n),a_{n+1})$ por lema.
		\end{itemize}
	\end{tcolorbox}
	Con lo anterior, entonces
	
	\begin{table}[h]
		\begin{tabular}{lcrp{6cm}}
			$mcm(ca_1, ..., ca_n, ca_{n+1})$ & $=\frac{|c^2 \cdot mcm(a_1,..,a_n) \cdot a_{n+1}|}{|c| \cdot mcd(mcm(a_1,...,a_n),a_{n+1})}$ & \ \ \ \ \\
			\\
			\ \ & $=\frac{|c||c| \cdot |mcm(a_1,..,a_n)\cdot a_{n+1}|}{|c| \cdot mcd(mcm(a_1,...,a_n), a_{n+1})}$ & Por propiedad de valor absoluto\\
			\\
			\ \ & $=\frac{ \mid (|c| \cdot mcm(a_1,...,a_n)) \cdot ca_{n+1} \mid }{mcd(|c| \cdot mcm(a_1,...,a_n),ca_{n+1})}$ & Por teorema\\
			\\
			\ \ & $=|c| \cdot \frac{|mcm(a_1,..,a_n) \cdot a_{n+1}|}{|mcd(mcm(a_1,...,a_n),a_{n+1}),}$ & Por hipótesis,$c \neq 0$ y por prop de mcd  \\
			\\
			\ \ & $=|c| \cdot mcm(mcm(a_1,..,a_n), a_{n+1})$ & Por teorema\\
			\\
			\ \ & $=|c| \cdot mcm(a_1,..,a_n, a_{n+1})_\blacksquare$ & Por propiedad de mínimo común múltiplo
		\end{tabular}
	\end{table}
	
	$\therefore$  $mcm(ca_1, ..., ca_n,a_{n+1}) = |c| \cdot mcm(a_1, ..., a_n, a_{n+1})$, por lo anterior, se cumple el paso inductivo.\\
	\\
	\textbf{$\therefore$ Es cierto que si $a_1, \dots, a_k$ es una colección de enteros no nulos, entonces\\
		$mcm(ca_1, ca_2, \dots, ca_k ) = |c|mcm(a_1, a_2, \dots, a_k )$ para $c \neq 0 \ y \ k \geq 2$.} $_\blacksquare$
	
	\vspace{1cm}
	
	%
	% Ejercicio 9
	%
	$\bullet$ \textbf{9.} Sean $a, b \in \mathbb{Z}$ no ambos nulos y $c \neq 0$. Muestra que $mcd(ca, cb) = |c|$ si y sólo si $mcd(a, b) = 1$.
	
	\subsubsection*{Respuesta}
	$\Rightarrow \rfloor$ Si $mcd(ca,cb) = |c|, entonces \ mcd(a.b)=1$\\
	\\
	Como $mcd(ca,cb) = |c|$, por propiedad sabemos que\\
	$mcd(ca,cb)= |c| \cdot mcd(a,b)$\\
	Por transitividad,\\
	$|c| = |c| \cdot mcd(a,b)$ y además como $c \neq 0$, entonces\\
	$1 = 1 \cdot mcd(a,b)$\\
	Asi, $mcd(a,b) = 1$ \\
	$\therefore$ Si $mcd(ca,cb) = |c|, entonces \ mcd(a.b)=1$ $_\blacksquare$\\
	\\
	$\Leftarrow \rfloor$ Si $ mcd(a.b)=1, \ entonces \ mcd(ca,cb) = |c|$\\
	\\
	Sabemos que  $ mcd(a.b)=1$, como $c \neq 0$, entonces \\
	$|c| = |c| \cdot mcd(a,b)$\\
	Por propiedad, \\
	$|c| \cdot mcd(a,b) = mcd(ca,cb) $\\
	Asi, por transitividad,\\
	$|c| = mcd(ca,cb)$\\
	$\therefore$ Si $ mcd(a.b)=1, \ entonces \ mcd(ca,cb) = |c|$ $_\blacksquare$\\
	\\
	Por definición de doble contención.\\
	Sean $a, b \in \mathbb{Z}$ no ambos nulos y $c \neq 0$, entonces $mcd(ca, cb) = |c|$ si y sólo si $mcd(a, b) = 1$ $_\blacksquare$
	\vspace{1cm}
	
	%
	% Ejercicio 10
	%
	$\bullet$ \textbf{10.} Muestra que si $p$ es un número primo y $k \in \{1, \dots, p-1 \}$, entonces $p | \binom{p}{k}$.
	
	\subsubsection*{Respuesta}
	Sea $p$ un número primo y $k \in \{1, \ldots, p - 1\}$, notemos que por propiedad de divisibilidad $p \mid p$. De lo anterior podemos afirmar que:
	
	\[ p \mid p \cdot \frac{(p - 1)!}{(k - 1)!((p-1) - (k-1))!} \]
	
	Ahora hacemos notar que $p > k$ y además son naturales, podemos afirmar que:
	
	\[ k \binom{p}{k} = p \binom{p-1}{k-1} \]
	
	Luego, podemos decir de igual forma que:
	
	\[ p \mid k \cdot \binom{p}{k} \]
	
	Notemos entonces que $p \mid k$ debido a que, como habíamos establecido por nuestra hipótesis $p > k$, por lo que este término no lo tomaremos en cuenta ya que es imposible que $p$ sea su divisor. Dicho esto concluimos que:
	
	\[ p \mid \binom{p}{k} _\blacksquare\]
	
	\vspace{1cm}
	%
	% Ejercicio 11
	%
	$\bullet$ \textbf{11.} Sean $a, b, t \in \mathbb{Z}$ con $t \neq 0$. Muestra que si $mcd(k, t) = 1$ y $at \equiv bt$ mod $k$, entonces $a \equiv b$ mod $k$.
	
	\subsubsection*{Respuesta}
	Como $at \equiv bt$ mod $k$, entonces por definición de congruencia \\
	$k \mid at - bt$, \\
	entonces $k \mid t (a-b)$\\
	\\
	Como $mcd(k,t)=1$ y $k \mid t (a-b)$, entonces por propiedad (\textit {Lema 2.2.8}), \\
	$k \mid a-b$, por definición de congruencia, entonces\\
	$a \equiv b$ mod $k$ $_\blacksquare$
	
	\vspace{1cm}
	%
	% Ejercicio 12
	%
	$\bullet$ \textbf{12.} Muestra que si $a \equiv b$ mod $k$, entonces $mcd(a, k) = mcd(b, k)$.
	
	\subsubsection*{Respuesta}
	\textbf{Hipotésis.} $a \equiv b$ mod $k$, entonces $k \mid a -b$\\
	\\
	Tenemos que $k \mid a$ y $k \mid b$.\\
	\\
	$mcd(a,k) \mid a$ y $mcd(a, k) \mid k$\\
	\\
	$mcd(b,k) \mid b$ y $mcd(b, k) \mid k$\\
	\\
	Como $mcd(a,k) \mid k$ y $k \mid b$, entonces $mcd(a, k) \mid b$\\
	\\
	Como $mcd(b,k) \mid k$ y $k \mid a$, entonces $mcd(b, k) \mid a$\\
	\\
	De lo anterior sabemos que $mcd(b,k) \mid a$ y  $mcd(b,k) \mid b$, tambien $mcd(a,k) \mid a$ y  $mcd(a,k) \mid b$\\
	\\
	$\therefore$ $mcd(a,k) = mcd(b,k)$ $_\blacksquare$
	
	\vspace{1cm}
	%
	% Ejercicio 13
	%
	$\bullet$ \textbf{13.} Sea $k = mcd(m, n)$. Muestra que si $a \equiv b$ mod $m$ y $c \equiv d$ mod $n$, entonces $a + c \equiv b + d$ mod $k$.
	
	\subsubsection*{Respuesta}
	\textbf{Hipotésis 1:} $a \equiv b$ mod $m$, entonces $m \mid a - b$\\
	\\
	\textbf{Hipotésis 2:} $c \equiv d$ mod $n$, entonces $n \mid c - d$\\
	
	Como $k = mcd(m,n)$, entonces $k \mid m$ y $k \mid n$\\
	
	Como $k \mid m$ y $m \mid a - b$, entonces:
	\[k \mid a - b\]
	Como $k \mid n$ y $n \mid c - d$, entonces:
	\[k \mid c - d\]
	Así:
	\[k \mid (a-b) + (c-d)\]
	\[= k \mid (a+c)-b-d\]
	\[= k \mid (a+c) - (c+d)\]
	\[a+c \equiv b+d \textit{ mod } k \text{ } _\blacksquare\]
	
	\vspace{1cm}
	%
	% Ejercicio 14
	%
	$\bullet$ \textbf{14.} Sean $a, b$ y $k$ enteros tales que $a \equiv b$ mod $k$. Muestra que si $0 \leq a < k$ y $0 \leq b < k$, entonces $a = b$.
	
	\subsubsection*{Respuesta}
	Sean a, b, k enteros tal que $a \equiv b$ mod $k$, por definición de congruencia \\
	$k \mid a-b$, por definición de divisibilidad \\
	existe $n \in \mathbb{Z}$ tal que $k \cdot n = a-b$ \\
	\\
	Como $0 \leq a < k$ y $0 \leq b < k$, entonces restando ambas desigualdades \\
	$0-k < a-b < k$, entonces\\
	$-k < a-b <k$ \\
	\\
	Dado que $a-b = k \cdot n $ y $k \cdot n \in \mathbb{Z}$, es decir, un múltiplo de k, pero sabemos que \\
	$-k < a-b <k$, por lo que $n=0$, entonces \\
	$a-b = k \cdot 0 = 0$ \\
	$a-b = 0$, despejando\\
	$a=b$ $_\blacksquare$
	
	\vspace{1cm}
	%
	% Ejercicio 15
	%
	$\bullet$ \textbf{15.} Considera la ecuación diofantina $56x + 378y= k$. Calcula todos los valores de $k$ entre 100 y 200 para
	los cuales dicha ecuación tiene solución entera. Calcula la solución para el caso en que $k = 154$.
	\subsubsection*{Respuesta}
	Para calcular los valores solicitados tenemos que sacar en primer lugar el $mcd(56,378)$ notemos entonces por algortimo de Euclides:\\
	$378=56 \cdot 6+42$\\
	$56=42 \cdot 1+14$\\
	$42=14 \cdot 3+0$\\
	Así, tomando el ultimo residuo distinto de 0, el $mcd(56,378)=14$\\
	\\
	Ahora para saber si $14 \mid k$ expresemos la combinación líneal de 14 respeto de 56 y 378\\
	Por lo anterior, supongamos que exiten $s,t \in \mathbb{Z}$ tales que $56t+378s=14$\\
	\\
	Tambien, debemos calcular un $r \in \mathbb{Z}$ tal que $14 \cdot r=k$\\
	Notemos que por hipotesis, k puede tomar valores entre 100 y 200, si sacamos los multiplos de 14 entre ese rango tenemos el caso particular de $14 \cdot 11 = 154$, asi $r=11 \ y \ k=154$\\
	
	
	\[ \text{mcd}(56, 378) = 14 \]
	
	Notemos entonces que para encontrar las soluciones enteras de la ecuación diofantina se debe cumplir que $\text{mcd}(56, 378) \mid K$ que entonces está en un rango $100 < k < 200$, notemos entonces que por definición todos los números enteros que sean múltiplos de 14 tendrán solución entera, obviamente en el rango impuesto:
	
	Ahora notemos que el único múltiplo de 14 que es mayor a 100 es:
	
	$14 \times 8 = 112$ de aquí lo único que resta para conseguir las soluciones es sumar a 112 de 14 en 14.
	
	\begin{align*}
		14 \times 9 &= 126 \\
		14 \times 10 &= 140 \\
		14 \times 11 &= 154 \\
		14 \times 12 &= 168 \\
		14 \times 13 &= 182 \\
		14 \times 14 &= 196
	\end{align*}
	
	Estos serán los únicos números para los cuales la ecuación diofantina tendrá soluciones enteras. Para el caso específico de $k = 154$, como ya sabemos que es divisor entre el mcd, entonces ahora sacamos la ecuación lineal tal que:
	
	\[ 56s + 378t = 14 \]
	
	esta será:
	
	\[ 56 \times 7 + 378 \times (-1) = 14 \]
	
	Notemos que $14 \times 11 = 154$
	
	por tanto para finalizar nuestra ecuación:
	
	\[ (11 \times 7, 11 \times -1) = (77, -11) \]
	
	Comprobando:
	
	\[ 56 \times 77 + 378 \times (-11) = 154 \]
	\vspace{1cm}
	
	\subsection*{Ejercicios extra}
	%
	% Ejercicio Extra 1
	%
	$\bullet$ \textbf{Extra 1.}Demuestra que todo número natural $n = p_1^{a_1} \cdots p_k^{a_k}$ no puede tener más de un factor primo $p_i$ mayor a $\sqrt{n}$.\\
	
	Como tenemos que demostrar una existencia única, procederemos por contradicción. Supongamos que existen dos factores primos $p_i > \sqrt{n}$ y $p_j > \sqrt{n}$ distintos.
	
	Dado que $p_i$ y $p_j$ son dos factores primos distintos, tenemos:
	\[ p_i \cdot p_j > \sqrt{n} \cdot \sqrt{n} = n \]
	
	De lo anterior, y considerando que por hipótesis tanto $p_i$ como $p_j$ son menores que $n$ (pues son factores primos de $n$), llegamos a una contradicción. Por definición de divisibilidad:
	\[ n \mid p_i \cdot p_j \]
	lo que implica por propiedades que:
	\[ n \geq p_i \cdot p_j > n \]
	
	Esto demuestra la contradicción $n > n$. De esto último concluimos que en la factorización:
	\[ n = p_1^{a_1} \cdots p_k^{a_k} \]
	no puede haber más de un factor primo $p_i$ mayor que $\sqrt{n}$.
	
	\vspace{1cm}
	
	%
	% ejercicio extra 2
	%
	$\bullet$ \textbf{Extra 2.} Sean $a_1,..., a_n$ una colección de enteros no nulos. Muestra que si $mcd(a_1,..., a_n)=1$, entonces se satisface que $mcm(a_1,..., a_n)=$ \[\prod_{i=1}^{n} a_i\]
	
	\subsubsection*{Respuesta}
	Sean $a_1,..., a_n$ una colección de enteros no nulos tales que $mcd(a_1,..., a_n)=1$. \\\\
	Por teorema, $mcm(a_1,..., a_n)=\frac{|a_1 \cdots a_n|}{mcd(a_1,..., a_n)}$, pero esto es lo mismo que: $mcm(a_1,..., a_n)=|a_1 \cdots a_n|$ ya que, $mcd(a_1,..., a_n)=1$.\\\\
	Entonces, como $mcm(a_1,..., a_n)\ge1$ por definición de mínimo común múltiplo, y $mcm(a_1,..., a_n)=|a_1 \cdots a_n|$ entonces, por definición de valor absoluto:\\\\
	$mcm(a_1,..., a_n)=a_1 \cdots a_n$\\\\
	$\therefore$ $mcm(a_1,..., a_n)=$\[\prod_{i=1}^{n} a_i\]\\\\
	$\therefore$ Se cumple que si $mcd(a_1,..., a_n)=1$, entonces se satisface que $mcm(a_1,..., a_n)=$ \[\prod_{i=1}^{n} a_i \ _\blacksquare \] 
	
\end{document}