\documentclass[12pt]{article}
\usepackage{graphicx}
\usepackage{caption}
\usepackage{subcaption}
\usepackage{tikz}
\usepackage{venndiagram}
\usepackage{venndiagram}
\usepackage{tcolorbox}
\usepackage{listings}
\usepackage{enumitem}
\usepackage{amsmath}
\usepackage{amssymb}
\usepackage{colortbl}
\usepackage{xcolor}
\usepackage[margin=1cm, top=1.5cm, bottom=1.5cm]{geometry}

\tcbuselibrary{breakable}

\title{\textbf{Algebra Superior II: Tarea 02}}
\author{Rendón Ávila Jesús Mateo}
\date{\today}

\begin{document}

\maketitle
\begin{center}
\vspace{2cm}
\includegraphics[width=0.19\textwidth]{Escudo.png}
\hspace{0.5cm}
\includegraphics[width=0.2\textwidth]{logo_ciencias.png}
\end{center}
\begin{center}
    \vspace{1cm}
    Facultad de Ciencias\\
    Universidad Nacional Autónoma de México\\
    \vspace{3cm}
    Profesor: Dr. Gerardo Miguel Tecpa Galván\\
\end{center}

\newpage

%
% Ejercicio 1
%
\textbf{1.} Sean $a$ un número par y $b$ un número impar. Muestra que $mcd(a, b)$ es impar.

\vspace{1cm}

%
% Ejercicio 2
%
\textbf{2.} Un grupo de 23 viajeros llega a un campamento y encuentra 63 montones de sacos, cada montón con
el mismo número de sacos, y un montón adicional con 7 sacos (en total hay 64 montones). Si sabemos que
los viajeros no podían cargar con más de 50 sacos cada uno y pudieron repartírselos por igual y sin abrirlos,
¿cuántos sacos había en cada uno de los montones?

\vspace{1cm}

%
% Ejercicio 3
%
\textbf{3.} Demuestra que si $a$ y $b$ son enteros no nulos, entonces $mcd(a, b) |mcm(a, b)$.

\vspace{1cm}

%
% Ejercicio 4
%
\textbf{4.} Muestra que si $p$ y $q$ son dos primos distintos, entonces para todo $a, b \in \mathbb{Z}^+$ se cumple que 
$mcd(p^a, q^b) = 1$.

\vspace{1cm}

%
% Ejercicio 5
%
\textbf{5.} Sean $a_1, \dots, a_n \in \mathbb{Z}$ una colección de enteros Muestra que si para todo $i, j \in \{1, . . . , k \}$ 
con $i \neq j$ se satisface que $mcd(a_i, a_j ) = 1$, entonces $mcd(a_k , a_1  \cdot a_2 \cdots a_{k-1}) = 1$.

\vspace{1cm}

%
% Ejercicio 6
%
\textbf{6.} Sean $a_1, \dots, a_n, b \in \mathbb{Z}$ una colección de enteros tales que para todo $i, j \in \{1, \dots, k\}$ con $i \neq j$
se satisface que $mcd(a_i, a_j ) = 1$. Muestra por inducción que si para todo $i \in \{1, \dots, k \}$ se cumple que $a_i |b$, entonces
$a_1 \cdot a2 \cdots a_k |b$.
\vspace{1cm}

%
% ejercicio 7
%
\textbf{7.} Sea $p$ un número primo. Muestra que si $k \in \mathbb{Z}$ es tal que $k < p$, entonces $p \nmid k!$

\vspace{1cm}

%
% Ejercicio 8
%
\textbf{8.} Sean $c \neq 0$ y $k \geq 2$. Muestra mediante inducción matemática que si $a_1, \dots, a_k$ es una colección de
enteros no nulos, entonces $mcm(ca_1, ca_2, \dots, ca_k ) = |c|mcm(a_1, a_2, \dots, a_k )$.
\vspace{1cm}

%
% Ejercicio 9
%
\textbf{9.} Sean $a, b \in \mathbb{Z}$ no ambos nulos y $c \neq 0$. Muestra que $mcd(ca, cb) = |c|$ si y sólo si $mcd(a, b) = 1$.

\vspace{1cm}

%
% Ejercicio 10
%
\textbf{10.} Muestra que si $p$ es un número primo y $k \in \{1, \dots, p-1 \}$, entonces $p | \binom{p}{k}$.

\vspace{1cm}
%
% Ejercicio 11
%
\textbf{11.} Sean $a, b, t \in \mathbb{Z}$ con $t \neq 0$. Muestra que si $mcd(k, t) = 1$ y $at \equiv bt$ mod $k$, entonces $a \equiv b$ mod $k$.

\vspace{1cm}
%
% Ejercicio 12
%
\textbf{12.} Muestra que si $a \equiv b$ mod $k$, entonces $mcd(a, k) = mcd(b, k)$.

\vspace{1cm}
%
% Ejercicio 13
%
\textbf{13.} Sea $k = mcd(m, n)$. Muestra que si $a \equiv b$ mod $m$ y $c \equiv d$ mod $n$, entonces $a + c \equiv b + d$ mod $k$.

\vspace{1cm}
%
% Ejercicio 14
%
\textbf{14.} Sean $a, b$ y $k$ enteros tales que $a \equiv b$ mod $k$. Muestra que si $0 \leq a < k$ y $0 \leq b < k$, entonces $a = b$.

\vspace{1cm}
%
% Ejercicio 15
%
\textbf{15.} Considera la ecuación diofantina $56x + 378y= k$. Calcula todos los valores de $k$ entre 100 y 200 para
los cuales dicha ecuación tiene solución entera. Calcula la solución para el caso en que $k = 154$.

\end{document}