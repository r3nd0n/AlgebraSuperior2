\documentclass[12pt]{article}
\usepackage{graphicx}
\usepackage{caption}
\usepackage{subcaption}
\usepackage{tikz}
\usepackage{venndiagram}
\usepackage{venndiagram}
\usepackage{tcolorbox}
\usepackage{listings}
\usepackage{enumitem}
\usepackage{amsmath}
\usepackage{amssymb}
\usepackage{colortbl}
\usepackage{xcolor}
\usepackage[margin=1cm, top=1.5cm, bottom=1.5cm]{geometry}

\tcbuselibrary{breakable}

\title{\textbf{Algebra Superior II: Tarea 01}}
\author{Rendón Ávila Jesús Mateo}
\date{\today}

\begin{document}
    Recordemos que $\overline{(a,b)}$ representa una solución a una ecuación y lo definimos $a - b$ como un numero en el conjunto de los enteros $\mathbb{Z}$.\\

    Cuando $a \leq b$, tenemos $\overline{(a,b)} = \overline{(0,k)} \Longrightarrow (a,b) \sim (0,k)$, $i.e$ $\overline{(0,k)}$ es un número negativo.\\

    Dada la relación $\sim$ definida en $\mathbb{Z}$
    \begin{align*}
        (a,b) &\sim (0,k)\\
        a + k &= b +0\\
        a + k &= b
    \end{align*}

    Así podemos obtner entonces la definición $a \leq b$\\

    $Def.$ Un elemento $a\leq b$ si existe un elmeento $k \in \mathbb{N}$ tal que $a + k = b$\\

    Lo mismo ocurre cuando $b \leq a$\\

    Es decir $\overline{(a,b)} = \overline{(k,0)} \Longrightarrow (a,b) \sim (k,0)$
    \begin{align*}
        (a,b) &\sim (k,0)\\
        a + 0 &= b + k\\
        a &= b + k
    \end{align*}

    Así podemos obtener la definición de $b \leq a$

\end{document}